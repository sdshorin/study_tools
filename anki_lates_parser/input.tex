\documentclass[a4paper,12pt,leqno]{article}

%%% Работа с русским языком
\usepackage{cmap}					% поиск в PDF
\usepackage{mathtext} 				% русские буквы в формулах
\usepackage[T2A]{fontenc}			% кодировка
\usepackage[utf8]{inputenc}			% кодировка исходного текста
\usepackage[english,russian]{babel}	% локализация и переносы

%%% Дополнительная работа с математикой
\usepackage{amsmath,amsfonts,amssymb,amsthm,mathtools} % AMS
\usepackage{icomma} % "Умная" запятая: $0,2$ --- число, $0, 2$ --- перечисление

%% Номера формул
%\mathtoolsset{showonlyrefs=true} % Показывать номера только у тех формул, на которые есть \eqref{} в тексте.
%\usepackage{leqno} % Нумерация формул слева

%% Свои команды
\DeclareMathOperator{\sgn}{\mathop{sgn}}

%% Перенос знаков в формулах (по Львовскому)
\newcommand*{\hm}[1]{#1\nobreak\discretionary{}
{\hbox{$\mathsurround=0pt #1$}}{}}

%%% Работа с картинками
\usepackage{graphicx}  % Для вставки рисунков
\graphicspath{{images/}{images2/}}  % папки с картинками
\setlength\fboxsep{3pt} % Отступ рамки \fbox{} от рисунка
\setlength\fboxrule{1pt} % Толщина линий рамки \fbox{}
\usepackage{wrapfig} % Обтекание рисунков текстом

%%% Работа с таблицами
\usepackage{array,tabularx,tabulary,booktabs} % Дополнительная работа с таблицами
\usepackage{longtable}  % Длинные таблицы
\usepackage{multirow} % Слияние строк в таблице

%%% Теоремы
\theoremstyle{plain} % Это стиль по умолчанию, его можно не переопределять.
\newtheorem{theorem}{Теорема}
\newtheorem{proposition}{Утверждение}[section]
\newtheorem{lemma}{Лемма}
 
\theoremstyle{definition} % "Определение"
\newtheorem{corollary}{Следствие}[theorem]
\newtheorem{definition}{Определение}[section]
\newtheorem{remark}{Замечание}

\renewenvironment{proof}{$\qed$}{{\qed}} %оформление доказательств
\renewcommand\qedsymbol{$\blacksquare$} %черный квадрат

%%% Программирование
\usepackage{etoolbox} % логические операторы

%%% Страница
\usepackage{extsizes} % Возможность сделать 14-й шрифт
\usepackage{geometry} % Простой способ задавать поля
	\geometry{top=25mm}
	\geometry{bottom=35mm}
	\geometry{left=10mm}
	\geometry{right=10mm}
	
%\usepackage{fancyhdr} % Колонтитулы
% 	\pagestyle{fancy}
 	%\renewcommand{\headrulewidth}{0pt}  % Толщина линейки, отчеркивающей верхний колонтитул
% 	\lfoot{Нижний левый}
% 	\rfoot{Нижний правый}
% 	\rhead{Верхний правый}
% 	\chead{Верхний в центре}
% 	\lhead{Верхний левый}
%	\cfoot{Нижний в центре} % По умолчанию здесь номер страницы

\usepackage{setspace} % Интерлиньяж
%\onehalfspacing % Интерлиньяж 1.5
%\doublespacing % Интерлиньяж 2
%\singlespacing % Интерлиньяж 1

\usepackage{lastpage} % Узнать, сколько всего страниц в документе.

\usepackage{soul} % Модификаторы начертания

\usepackage{hyperref}
\usepackage[usenames,dvipsnames,svgnames,table,rgb]{xcolor}
\hypersetup{				% Гиперссылки
    unicode=true,           % русские буквы в раздела PDF
    pdftitle={Заголовок},   % Заголовок
    pdfauthor={Автор},      % Автор
    pdfsubject={Тема},      % Тема
    pdfcreator={Создатель}, % Создатель
    pdfproducer={Производитель}, % Производитель
    pdfkeywords={keyword1} {key2} {key3}, % Ключевые слова
    colorlinks=true,       	% false: ссылки в рамках; true: цветные ссылки
    linkcolor=red,          % внутренние ссылки
    citecolor=black,        % на библиографию
    filecolor=magenta,      % на файлы
    urlcolor=cyan           % на URL
}

\usepackage{csquotes} % Еще инструменты для ссылок

\usepackage[style=authoryear,maxcitenames=2,backend=biber,sorting=nty]{biblatex}

\usepackage{multicol} % Несколько колонок

\usepackage{tikz} % Работа с графикой
\usepackage{pgfplots}
\usepackage{pgfplotstable}
\title{Доказательства для коллоквиума I по дискретной математике, КНАД  \\ \small Version 2}
\begin{document}
\maketitle

\begin{theorem}
Основные свойства логических связок.

Идемпотентность:

\quad $ p \wedge p \equiv p$

\quad  $ p \vee p \equiv p$

Двойное отрицание:

\quad  $\neg (\neg p) \equiv p $

Законы де Моргана:

\quad  $\neg(p \vee q) \equiv \neg p \wedge \neg q $

\quad  $\neg(p \wedge q) \equiv \neg p \vee \neg q $

Коммутативность:

\quad  $ p \vee q \equiv q \vee p$

\quad  $ p \wedge q \equiv q \wedge p$

Ассоциативность:

\quad $ p \wedge (q \wedge r) \equiv (p \wedge q) \wedge r$

\quad  $ p \vee (q \vee r) \equiv (p \vee q) \vee r$

Дистрибутивность:

\quad  $ p \wedge (q \vee r) \equiv (p \wedge q) \vee (p \wedge r) $

\quad  $ p \vee (q \wedge r) \equiv (p \vee q) \wedge (p \vee r) $

Контрапозиция:

\quad  $p \rightarrow q \equiv (\neg q)\rightarrow (\neg p)   $

\end{theorem}
\begin{proof}

Все формулы доказываются через вычисление табицы истинности для левой и правой части. Например, для закона де Моргана $\neg(p \vee q) \equiv \neg p \wedge \neg q $:

\begin{tabular}{cc|c|c}
p & q & $\neg(p \vee q) $ & $ \neg p \wedge \neg q$\\
\hline
0 & 0 & 1 & 1\\
0 & 1 & 0 & 0\\
1 & 0 & 0 & 0\\
1 & 1 & 0 & 0\\
\end{tabular}

Так как при всех возможных p и q $\neg(p \vee q) $  и $ \neg p \wedge \neg q$ принимают одинаковые значения, эти выражения эквивалентны.

\end{proof}

\begin{theorem}
	Основные свойства операций над множествами.
	
	Пусть $A,B,C$ суть подмножества универсума  $\mathbb{U}$. Тогда справедливо:
	\begin{enumerate}
		\item $A\cap{}A=A$\\
		$A\cup{}A=A$\\
		$A\backslash{}A=\varnothing$\\
		$A\bigtriangleup{}A=\varnothing$
		\item $(A')'=A$
		\item Законы де Моргана:\\
		\quad  $(A\cup{}B)'=A'\cap{}B'$\\
		\quad $(A\cap{}B)'=A'\cup{}B'$
		\item Коммутативность:\\
		\quad $A\cap{}B=B\cap{}A$\\
		\quad $A\cup{}B=B\cup{}A$\\	
		\quad $A\bigtriangleup{}B=B\bigtriangleup{}A$
		\item Ассоциативность:\\
		\quad $(A\cap{}B)\cap{}C=A\cap{}(B\cap{}C)$\\
		\quad $(A\cup{}B)\cup{}C=A\cup{}(B\cup{}C)$
		\item Дистрибутивность:\\
		\quad $A\cap{}(B\cup{}C)=(A\cap{}B)\cup{}(A\cap{}C)$\\
		\quad $A\cup{}(B\cap{}C)=(A\cup{}B)\cap{}(A\cup{}C)$
		\item Свойства тождества:\\
		\quad $A\cup{}\varnothing=A$\\
		\quad $A\cap{}$$\mathbb{U}$$\,=A$
		\item Свойства дополнения:\\
		\quad $A\cup{}A'=\,$$\mathbb{U}$\\
		\quad $A\cap{}A'=\varnothing$
		\item $A\backslash{}B=A\cap{}B'$\\
		$A\bigtriangleup{}B=(A\backslash{}B)\cup{}(B\backslash{}A)$
	\end{enumerate}
\end{theorem}
\begin{proof}

Первые два верны по определению, остальные доказываются с помощью переходов в алгебре логики по определению операций над множествами.

\end{proof}


\begin{theorem}
Правило произведения.

Пусть $A$, $B$ --- конечные множества. Тогда $|A \times B| = |A| \cdot |B|$.
\end{theorem}

\begin{proof}

Будем доказывать с помощью матиндукции по мощности множества $A$, множество $B$ считаем конечным: $|B| = k$.

\begin{itemize}
    \item База: $|A| = n = 1$. Тогда $A \times B = \{(a, b_1), (a, b_2),\dots, (a, b_k)\}$
    
    Очевидно, что $|A \times B| = 1 \cdot |B| = |A| \cdot |B|$
    
    \item Предположение: Пусть $|A| = n$ и $|A \times B| = |A| \cdot |B|$
    
    \item Шаг: Пусть $A = \{a_1, a_2,\dots,a_{n+1}\}$
    
    Рассмотрим два множества:
    
    \begin{enumerate}
        \item $C = \{(a, b): (a \in A \setminus \{a_{n+1}\}) \wedge (b \in B)\} = \tilde{A} \times B$, где $\tilde{A} = A \setminus \{a_{n+1}\}$
        
        \item $D = \{(a, b): (a = a_{n+1}) \wedge (b \in B)\}$
    \end{enumerate}
    
    Очевидно, что:
    
    \begin{itemize}
        \item $C \cap D = \varnothing$
        
        \item $|C| = |\tilde{A} \times B| = |\tilde{A}| \cdot |B| = n \cdot k$ (в силу предположения)
        
        \item $|D| = k$ (в силу базы)
        
        \item $A \times B = C \cup D$
    \end{itemize}
    
    $|A \times B| = |C \cup D| = |C| + |D| = n \cdot k + k = (n+1) \cdot k = |A| \cdot |B|$
    
\end{itemize}
\end{proof}

\begin{theorem}

\textit{О числе слов в конечном алфавите.}

Пусть $A$ --- конечное множество. Тогда $|A^n| = |A|^n$.
\end{theorem}

\begin{proof}

Доказательство по индукции по длине слова $n$.

\begin{itemize}
    \item База: $n = 1$. $A^1 = \{(a): a \in A\}$. Тогда $|A^1| = |A| = |A|^1$
    
    \item Предположение: $|A^n| = |A|^n$
    
    \item Шаг: Докажем, что $|A^{n+1}| = |A \times A^n|$
    
    Построим биекцию между ними. Элемент множества $A \times A^n$ имеет вид $(a, \alpha)$, где $a \in A$, $\alpha = (a_1, \dots, a_n) \in A^n$, а элемент $A^{n+1}$ имеет вид $(a, a_1, \dots, a_n)$.
    
    Заметим, что отображение $(a, \alpha) \mapsto (a, a_1,\dots,a_n)$ является биекцией
    
    \begin{enumerate}
        \item Инъективность: пусть образы равны, то есть $(a_1, a_1',\dots,a_n') = (a_2, a_1'',\dots,a_n'')$. Из определения последовательности $a_1 = a_2$ и $a_i' = a_i''$ $\forall i \in \{1,\dots,n\}$, то есть $a_1 = a_2$ и $\alpha_1 = \alpha_2$
        
        \item Сюръективность: для любого образа $(a_0, a_1, \dots, a_n) \in A^{n+1}$ существует прообраз $(a, \alpha) \in A \times A^n$, $a = a_0, \alpha = (a_1, \dots, a_n)$ 
    \end{enumerate}
    
    Теперь $n = n + 1$. $|A^{n+1}| = |A \times A^n| = |A| \cdot |A^n| = |A| \cdot |A|^n = |A|^{n+1}$.
\end{itemize}
\end{proof}

\begin{theorem}

\textit{Формула включений - исключений.}

Для любых конечных множеств $A_1,\dots,A_n$ верна формула
\[|A_1 \cup \dots \cup A_n| = \sum\limits_i |A_i| - \sum\limits_{i < j} |A_i \cap A_j| + \sum\limits_{i < j < k} |A_i \cap A_j \cap A_k| - \sum\limits_{i < j < k < l} |A_i \cap A_j \cap A_k \cap A_l| + \dots + (-1)^{n+1} |A_1 \cap \dots \cap A_n|\]
\end{theorem}

\begin{proof}

\begin{lemma}
Докажем, что $\forall m \in \mathbb{N}$ справедливо равенство
\[C_m^1 - C_m^2 + C_m^3 - \dots + (-1)^{m+1}C_m^m = 1\]

\begin{proof}

Воспользуемся биномом Ньютона:

$0 = (1 - 1)^m = \sum\limits_{k=0}^m C_m^k \cdot 1^{m-k} \cdot (-1)^k = 1 + \sum\limits_{k=1}^m C_m^k \cdot (-1)^k = 1 - \sum\limits_{k=1}^m C_m^k \cdot (-1)^{k+1}$

Откуда $1 = \sum\limits_{k=1}^m C_m^k \cdot (-1)^{k+1} = C_m^1 - C_m^2 + C_m^3 - \dots + (-1)^{m+1}C_m^m $

\end{proof}
\end{lemma}

Теперь возьмем произвольный элемент $x \in A_1 \cup \dots \cup A_n$. В формуле нам нужно доказать, что этот элемент встречается один раз. Без ограничения общности считаем, что $x$ принадлежит $A_1,\dots,A_m$ и не принадлежит $A_{m+1},\dots,A_n$.

\begin{itemize}
    \item в первой сумме $\sum\limits_i |A_i|$ элемент $x$ посчитан ровно $C_m^1$ раз (в множествах $A_1,\dots,A_m$)
    
    \item во второй сумме $\sum\limits_{i < j} |A_i \cap A_j|$ элемент $x$ посчитан ровно $C_m^2$ раз (количество попарных пересечений)
    
    $\dots$
    
    \item в $m$-й cумме $\sum\limits_{i_1 < i_2 < \dots < i_m} |A_{i_1} \cap A_{i_2} \cap \dots \cap A_{i_m}|$ элемент $x$ посчитан $C_m^m = 1$ раз
    
    \item в $m+1$-й сумме $x$ не учитывается, так как он не принадлежит всем множествам.
\end{itemize}

Тогда элемент $x$ посчитан $C_m^1 - C_m^2 + C_m^3 - \dots + (-1)^{m+1}C_m^m$ раз. 

По лемме эта сумма равна 1.
\end{proof}

\begin{theorem}\textit{Основные свойства областей достижимости.}

Для любого графа и вершин $v_1, v_2, v_3$ справедливо:

\begin{enumerate}
    \item Для любого графа и вершин $v_1, v_2, v_3$ справедливо:
    
    \begin{itemize}
        \item Рефлексивность: $v_1 \in C(v_1)$
    
        \item Cимметричность: $v_1 \in C(v_2) \iff v_2 \in C(v_1)$
    
        \item Транзитивность: $\left( v_1 \in C(v_2) \right)$ и $\left( v_2 \in C(v_3) \right) \implies v_1 \in C(v_3)$
    \end{itemize}

    \item Пусть $C(v_1)$ и $C(v_2)$ области достижимости вершин $v_1$ и $v_2$. Тогда $C(v_1) \cap C(v_2) = \varnothing$ или $C(v_1) = C(v_2)$
\end{enumerate}

\end{theorem}

\begin{proof}

\begin{enumerate}

    \item  $v_1$ ---  по определению есть путь длины 0 из вершины $v_1$ в вершину $v_1$, тогда первое утверждение верно
    \item Если $v_1, u_1, u_2, \dots, u_k, v_2$ --- путь в графе, то и $v_2, u_k, \dots, u_2, u_1, v_1$ - тоже путь. Тогда $v_1 \in C(v_2), v_2 \in C(v_1)$ эквивалентны
    \item Так как $\left( v_1 \in C(v_2) \right) \wedge \left( v_2 \in C(v_3) \right)$, то существуют пути $v_2, u_1, u_2, \dots, v_1$ и $v_3, w_1, w_2, \dots, v_2$. Тогда $v_3, w_1, w_2, \dots, v_2, u_1, u_2, \dots, v_1 $ - тоже путь, что и означает, что $v_1 \in C(v_3)$
    
    \item Если $C(v_1)$ и $C(v_2)$ не пересекаются, то все доказано. Пусть пересекаются и $w \in C(v_1) \cap C(v_2)$. Если $x \in C(v_1)$, то существуют пути из $x$ в $v_1$, из $v_1$ в $w$ и из $w$ в $v_2$. Значит, что существует путь из $x$ в $v_2$, то есть $C(v_1) \subseteq C(v_2)$. Аналогично $C(v_2) \subseteq C(v_1)$. 
\end{enumerate}

\end{proof}

\begin{theorem} \textit{О наличии простого пути.}

Если вершины $x$ и $y$ связаны в графе $G$, то в этом графе найдется простой путь с началом в $x$ и концом в $y$.

\end{theorem}

\begin{proof}

Пусть вершины $x$ и $y$ связаны, то есть множество путей из $x$ в $y$ не пустое $M \neq \varnothing$. Возьмем кратчайший путь из этого множества (если все пути одной длины, то возьмём любой):
\[x = v_1, v_2,\dots,v_n = y\]

Пусть этот путь не является простым, то есть $\exists \, i, j : i < j : v_i = v_j$

\begin{itemize}
    \item если $j = n$, то путь $x = v_1, v_2,\dots,v_i = v_n = y$ имеет длину $i - 1 < n - 1$, что противоречит тому, что путь самый короткий
    
    \item если $j \neq n$, то путь $x = v_1, v_2, \dots, v_i, v_{j+1},\dots, v_{n-1}, v_n = y$ имеет длину строго меньшую длину, чем у выбранного нами, что противоречит выбору кратчайшего пути.
\end{itemize}

\end{proof}

\begin{theorem}\textit{Об описании лесов.}

Пусть граф $G$ --- простой неориентированный граф. Тогда следующие утверждения эквивалентны:

\begin{enumerate}
    \item Каждое ребро графа $G$ является мостом
    
    \item Для любых связанных вершин $x$ и $y$ в графе $G$ существует единственный простой путь из $x$ в $y$
    
    \item В графе $G$ нет циклов
\end{enumerate}

\end{theorem}

\begin{proof}

\begin{itemize}
    \item Докажем $1 \implies 2$, что эквивалентно $\neg 2 \implies \neg 1$
    
    Пусть $x = v_1, v_2, \dots, v_n = y$ и $x = u_1, u_2, \dots, u_k = y$ - различные простые пути, соединяющие вершины $x$ и $y$.
    
    Тогда $\exists \, i: v_1 = u_1, \dots, v_{i - 1} = u_{i - 1}$ и $v_i \neq u_i$ (если не существует, то пути совпадают)
    
    \includegraphics[scale=0.3]{Дискра/Коллок 1/Об описании лесов.png}
    
    При удалении ребра $\{v_{i-1}, v_i\}$ вершины $v_{i-1}$ и $v_i$ остаются связанными путем $v_{i-1} ,u_i, \dots, u_k = y = v_n, v_{n-1}, \dots, v_i$, тогда это ребро не является мостом.
    
    \item $2 \implies 3 \iff \neg 3 \implies \neg 2$
    
    Пусть $v_1, v_2, \dots, v_n, v_1$ - цикл в графе $G$. Тогда вершины $v_1$ и $v_n$ связаны в графе $G' = G \setminus \{\{v_n, v_1\}\}$, но тогда между $v_1$ и $v_n$ существует простой путь в $G'$ и этим же путем связаны в графе $G$. Но в $G$ эти вершины связаны еще и путем длины 1 $\{v_1, v_n\}$. Эти пути различны.
    
    \item $3 \implies 1 \iff \neg 1 \implies \neg 3$
    
    Пусть $\{x, y\}$ не является мостом в графе $G$. Тогда в $G' = G \setminus \{\{x, y\}\}$ вершины $x$ и $y$ остаются связанными и существует простой путь в $G'$ $x = v_1, v_2, \dots, v_n = y$ (все вершины различны и длина больше 1). Тогда $x = v_1, v_2, \dots, v_n = y, x$ - цикл в $G$. Ребра различны и вершин не меньше~3.
\end{itemize}

\end{proof}

\begin{theorem}\textit{О размерности графа.}

\begin{enumerate}
    \item Пусть графы $G$ и $G'$ отличаются лишь тем, что ребро $\{x, y\}$ присутствует в $G'$ и отсутствует в $G$: $V = V'$ и $E' = E \cup \{\{x, y\}\}$. Тогда:
    
     \begin{equation*}
         \dim(G') =  \left\{
            \begin{aligned}
                & \dim(G), \text{ если вершины } x \text{ и  } y \text{ находятся в разных компонентах связности в графе } G \\
                & \dim(G) + 1, \text{ если вершины } x \text{ и  } y \text{ находятся в одной компоненте связности в графе } G
            \end{aligned}\right.
     \end{equation*}
    
    
    \item Для любого графа $G$ $\dim(G) \geq 0$
\end{enumerate}

\end{theorem}

\begin{proof}
    
    \begin{enumerate}
        \item Пусть вершины $x$ и $y$ лежат в разных компонентах связности графа $G$. $C(v), C'(v)$ - области достижимости вершины $v$ в графе $G$ и $G'$ соответственно.
    
        Должно быть видно, что $C'(x) = C'(y) = C(x) \cup C(y)$
    
        \includegraphics[scale=0.2]{Дискра/Коллок 1/О размерности графа.png}
    
        Области достижимости вершин из других компонент связности остаются прежними.
    
        Получается: $n' = n, m' = m + 1, p' = p - 1$ ($n$ - вершины, $m$ - ребра, $p$ - компоненты связности)
    
        $\dim(G') = -n' + m' + p' = -n + m + 1 + p - 1 = -n + m + p = \dim(G)$
    
        Теперь пусть вершины лежат в одной компоненте связности графа $G$. Тогда добавление ребра $\{x, y\}$ не изменяет количество компонент связности, поэтому $n' = n, m' = m + 1, p' = p$ и $\dim(G') = -n' + m' + p' = -n + m + 1 + p = -n + m + p + 1 = \dim(G) + 1$
        
        \item Доказательство по индукции по числу ребер.
        
        \begin{itemize}
            \item База: $m = 0$. $\dim(G) = -n + 0 + p = -n + n = 0$
            
            \item Предположение: Пусть для графа на $m$ ребрах верно утверждение
            
            \item Шаг: Рассмотрим граф $G'$ с $m + 1$ ребром. Если мы уберем одно ребро, то получится граф $G$ с $m$ рёбрами. В силу предположения индукции его размерность больше либо равна 0. А в силу прошлого утверждения от добавления ребра размерность графа не уменьшается. Значит и размерность графа с $m + 1$ ребром больше либо равна 0.
        \end{itemize}
    \end{enumerate}
    
    \end{proof}

\begin{theorem}\textit{О размерности леса.}

Пусть $G$ - простой неориентированный граф. Тогда следующие утверждения эквивалентны:

\begin{itemize}
    \item[1)] Каждое ребро графа $G$ является мостом
    
    \item[2)] $\dim(G) = 0$
\end{itemize}

\end{theorem}

\begin{proof}
    
    \begin{enumerate}
        \item $2 \implies 1$.
        Пусть размерность 0, но в графе есть ребро, которое не мост. Удалим это ребро и получим граф $G'$, причем $n' = n, m' = m - 1, p' = p$. Тогда $\dim(G') = -n' + m' + p' = -n + m - 1 + p = 0 - 1 < 0$. Противоречие.
        
        \item $1 \implies 2$.
        Докажем по индукции по числу ребер в графе $G$.
        
        \begin{itemize}
            \item База: $m = 0$. $\dim(G) = -n + 0 + p = -n + n = 0$
            
            \item Предположение: пусть для любого графа, в котором ровно $m$ рёбер и каждое ребро является мостом, утверждение верно
            
            \item Шаг: Пусть в графе $G'$, где $m' = m + 1$ и все ребра мосты, а также $G$ - граф, полученный из графа $G'$ удалением ребра. Тогда $\dim(G') = \dim(G) = 0$, так как количество вершин одинаковое, в $G'$ ребер на одно больше, но количество компонент связности на одно меньше, чем в $G$.
        \end{itemize}
        
    \end{enumerate}
    
\end{proof}

\begin{theorem}\textit{О листьях в дереве.}

Если граф $G$ - дерево, имеющее хотя бы две вершины, то в нем существует хотя бы два листа.

\end{theorem}

\begin{proof}

Пусть в дереве $G$ $n$ вершин, тогда ребер $n - 1$ (из размерности леса). Пусть степени каждой вершины $d_1, d_2, \dots, d_n$. Тогда
\[\sum\limits_{i = 1}^n d_i = 2(n - 1)\]

\[\sum\limits_{i = 1}^n (d_i - 2) = -2\]

Так как граф связный и $n \geq 2$, то $\forall \, i \in \{1, 2, \dots, n\}$ $d_i \geq 1$, что означает, что $d_i - 2 \geq -1$. В силу равенства выше, существуют хотя бы два номера $i$ и $j$: $d_i - 2 = -1$ и $d_j - 2 = -1$, иначе вся сумма будет не меньше $-1$.
\end{proof}

\begin{theorem}\textit{Об остовных деревьях.}

У любого связного графа существует остовное дерево.

\end{theorem}

\begin{proof}

Пусть в графе есть ребра, которые не являются мостами. Будем удалять эти ребра по одному. В результате удаления граф все так же будет связным, так как удаляем не мосты. Эта процедура конечна, так как множество ребер конечно. В итоге мы получим остовное дерево, так как количество вершин останется прежним.

\end{proof}

\begin{theorem}\textit{Основные свойства компонент сильной связности.}

Пусть $G(V, E)$ - орграф. Тогда $\forall x, y, z \in V$ справедливо:

\begin{itemize}
    \item Рефлексивность: $x \in C(x)$
    
    \item Симметричность: $x \in C(y) \iff y \in C(x)$
    
    \item Транзитивность: $x \in C(y) \wedge y \in C(z) \implies x \in C(z)$
    
    \item $w \in C(x) \cap C(y) \implies C(x) = C(y)$ (компоненты сильной связности задают разбиение множества $V$)
\end{itemize}

\end{theorem}

\begin{proof}

\begin{itemize}
    \item Первые два очевидны и доказываются по определению
    
    \item Если $x \in C(y)$, то существует ориентированные пути $x = v_0, v_1, \dots, v_n = y$ и $y = u_0, u_1, \dots, u_k = x$. Если $y \in C(z)$, то также существуют ориентированные пути $y = \tilde{v}_0, \tilde{v}_1, \dots, \tilde{v}_n = z$ и $z = \tilde{u}_0, \tilde{u}_1, \dots, \tilde{u}_k = y$. Тогда пути $x = v_0, v_1, \dots, v_n = y = \tilde{v}_0, \tilde{v}_1, \dots, \tilde{v}_n = z$ и $z = \tilde{u}_0, \tilde{u}_1, \dots, \tilde{u}_k = y = u_0, u_1, \dots, u_k = x$ - ориентированные пути из $x$ в $z$ и обратно.
    
    \item Покажем, что $w \in C(x) \implies C(w) = C(v)$.
    
    Возьмем произвольную вершину $v \in C(x)$. Так как $w \in C(x)$, то $x \in C(w)$. По транзитивности получаем, что $v \in C(w)$ и так как $v$ - произвольная вершина из $C(x)$, то $C(x) \subseteq C(w)$.
    
    Возьмем произвольную вершину $v \in C(w)$. Так как $w \in C(x)$, то по транзитивности $v \in C(x)$. Так как $v$ - произвольная вершина, то $C(w) \subseteq C(x)$.
    
    Получаем, что $C(x) = C(w)$. Аналогично получаем, что $C(y) = C(w)$
\end{itemize}

\end{proof}

\begin{theorem}\textit{О вершинах нулевой степени входа и нулевой степени выхода.}

Пусть $G(V, E)$ - орграф без ориентированных циклов. Тогда:

\begin{enumerate}
    \item[1)] $\exists v \in V: \; indeg(v)=0$
    \item[2)] $\exists v \in V: \; outdeg(v)=0$
\end{enumerate}

\end{theorem}

\begin{proof}

2) Пусть такой вершины нет, то есть $\forall v \in V: outdeg(v) \geq 1$. Рассмотрим произвольную вершину $x \in V$: так как $outdeg(x) \geq 1$, то из $x$ исходит хотя бы одно ребро. Перейдем по такому ребру, скажем в вершину $y$, затем удалим ребро, по которому мы попали из $x$ в $y$. 

Если из $y$ исходит ребро, то повторим описанный выше процесс (переходим из $y$ в, скажем $z$, и удаляем ребро). Будем продолжать этот процесс пока это возможно. Мы не сможем сделать следующий шаг тогда и только тогда, когда из вершины, в которой мы находимся, пусть это будет вершина $w$, не исходит ни одного ребра (если есть исходящее из $w$ ребро, то мы по нему перейдем). 

Так как мы предположили, что $outdeg(v) \geq 1$ для всех вершин исходного графа, то это значит, что в вершину $w$ мы пришли повторно. Значит по определению ориентированного цикла в исходном графе есть цикл (содержащий вершину $w$), а это противоречит условию. Значит предположение неверно, то есть $\exists v \in V: \; outdeg(v) = 0$. \\

1) Немедленно следует из пункта 2: достаточно рассмотреть граф, полученный из исходного обращением ориентации всех рёбер. 

\end{proof}

\begin{theorem}\textit{Описание ациклических орграфов.}

Пусть $G(V, E)$ - орграф без петель. Тогда следующие утверждения эквивалентны.

\begin{itemize}
    \item[1)] В орграфе $G$ каждая компонента сильной связности состоит из одной вершины
    
    \item[2)] В орграфе $G$ нет ориентированных циклов
    
    \item[3)] Множество вершин $V$ можно пронумеровать натуральными числами от 1 до $|V|$ так, чтобы каждое ребро было направлено из вершины с меньшим номеров в вершину с большим номером
\end{itemize}

\end{theorem}

\begin{proof}

\begin{itemize}
    \item $1 \implies 2 \iff \neg 2 \implies \neg 1$.
    
    Пусть в орграфе без петель есть ориентированный цикл. Тогда в этом цикле как минимум две вершины (если одна, то это петля). Эти две вершины сильно связаны и находятся в одной компоненте сильной связности.
    
    \item $2 \implies 3$.
    
    Будем доказывать по индукции по числу вершин.
    
    \begin{enumerate}
        \item База: $|V| = n = 1$. Пронумеруем единственную вершину номером 1. Условие выполнено
        
        \item Предположение: для $|V| = n - 1$ утверждение справедливо
        
        \item Шаг: $|V| = n$ и в нем нет циклов. Тогда $\exists v \in V: outdeg(v) = 0$. Удалим эту вершину~$v$ со всеми входящими из нее ребрами. Тогда полученный граф на $n - 1$ вершине удовлетворяет предположению индукции. Тогда пронумеруем вершины от 1 до $n - 1$ так, что все ребра <<идут вверх>>. Вернём вершину $v$ и присвоим ей номер $n$. Все ребра все еще <<идут вверх>>, так как мы добавили ребра вида $\{x, v\}$, где $v$ - вершина с наибольшим номером.
        
    \end{enumerate}
    
    \item $3 \implies 1$.
    
    Пусть импликация неверна, то есть вершины можно промеровать, но в орграфе есть компоненты сильной связности с не менее двумя вершинами $x$ и $y$.
    
    Будем считать, что номер $x$ меньше номера $y$. Так как они находятся в одной компоненте связности, то существует ориентированный путь, ведущий из $y$ в $x$. Так как ребра  <<идут вверх>>, то начало пути $y$ имеет меньший номер, чем конец пути $x$. Противоречие. 
\end{itemize}

\end{proof}

\begin{theorem} \textit{Описание эйлеровых орграфов.}

	Пусть $G(V,E)$ есть орграф без изолированных вершин. Тогда следующие условия эквивалентны:
	
	\begin{enumerate}
		\item В орграфе $G(V,E)$ есть эйлеров цикл
		\item Орграф $G(V,E)$ сильно связен и у любой его вершины степени входа и выхода равны
		
		(т.е. $\forall v \in V: indeg(v)=outdeg(v)$).
	\end{enumerate}

\end{theorem}

\begin{proof}

\begin{itemize}
    \item $1 \implies 2$.
    
    Так как эйлеров цикл существует, то он проходит через все вершины (здесь мы пользуемся тем, что изолированных вершин нет). Значит, идя по этому циклу, можно дойти от любой вершины до любой другой. В силу определения сильной связности орграфа, это значит, что орграф $G(V,E)$ сильно связен.
    
    Рассмотрим произвольную вершину $v \in V$. Пусть эта вершина встречается в эйлеровом цикле $k$ раз. Это значит, что, двигаясь по эйлеровому циклу, мы приходим в $v$ и уходим из $v$ ровно $k$ раз. Так как в эйлеровом цикле рёбра не повторяются, то это значит, что у вершины $v$ есть (по крайней мере) $k$ входящих и $k$ исходящих рёбер. Так как эйлеров цикл проходит по всем рёбрам орграфа, то других рёбер у вершины $v$ быть не может. Значит $indeg(v) = outdeg(v) = k$.
    
    \item $2 \implies 1$.
    
    Пусть орграф $G$ сильно связен и у каждой его вершины степень входа равна степени выхода. В орграфе $G$ рассмотрим всевозможные ориентированные пути, которые не проходят дважды по одному ребру. Так как в $G$ есть хотя бы одно ребро (иначе все вершины изолированные), то множество таких путей не пусто. ыберем из множества всех таких путей путь с самым большим количеством вершин (если есть несколько таких путей, то возьмем любой), пусть это путь $\tau$:
    \[
    \tau: \; (v_1, v_2, \ldots, v_{t-1}, v_t)
    \]
    (длина $\tau$ конечна и не превышает числа рёбер $|E|$) \\
    Сначала покажем, что $v_1=v_t$, то есть что путь $\tau$ - это ориентированный цикл. Так как $\tau$ это путь с самым большим количеством вершин, то к нему нельзя добавить никакое ребро вида ($v_t, u$) (иначе мы добавим новую вершину~$u$, ведь по определению этот путь не проходит дважды по одному ребру). Это значит, что все исходящие из $v_t$ рёбра уже входят в $\tau$. Последнее возможно только если $v_1=v_t$. Действительно, пусть $v_1 \not= v_t$ и вершина $v_t$ входит в путь $\tau$ ровно $k$ раз:
    \[
    v_1, \ldots, v_t, \ldots, v_t, \ldots, v_t, \ldots, v_t
    \]
    причём первая вершина не равна $v_t$. Тогда в вершину $v_t$ входит как минимум $k$ рёбер и выходит ровно $k - 1$ ребро (ведь всех выходящие из $v_t$ рёбра присутствуют в пути $\tau$). Это противоречит тому, что $\mathrm{outdeg}(v_t) = \mathrm{indeg}(v_t)$. Значит $v_1 = v_t$ и $\tau$ - это ориентированный цикл.
    
    Осталось показать, что $\tau$ - это эйлеров цикл, то есть что все рёбра орграфа $G$ входят в $\tau$:
    
    \begin{itemize}
        \item Если в каждой вершине цикла $\tau$ использованы все рёбра, то из вершин цикла $\tau$ нельзя попасть в вершины, не принадлежащие $\tau$. Это значит, что все вершины орграфа $G$ присутствуют в цикле $\tau$ (здесь мы пользуемся тем, что орграф $G$ сильно связен). Следовательно в $\tau$ присутствуют и все рёбра орграфа $G$ (так как мы предположили, что в каждой вершине цикла $\tau$ использованы все рёбра и доказали, что все вершины орграфа $G$ присутствуют в $\tau$). 
    
    \item Предположим, что у некоторой вершины $v_i$ из цикла $\tau$ есть ребро вида ($v_i, u$), не     входящее в $\tau$. Тогда путь:
    \[
    \tau': \; (v_{i+1}, \ldots, v_t=v_1, \ldots, v_i, u)
    \]
    содержит на 1 вершину больше чем путь $\tau$, что противоречит выбору $\tau$ как пути с самым большим количеством вершин. Значит рёбер вида ($v_i, u$), не входящих в $\tau$ нет.
    
    \item Предположим, что у некоторой вершины $v_i$ из цикла $\tau$ есть ребро вида ($u, v_i$), не входящее в $\tau$. Тогда путь:
    \[
    \tau'': \; (u, v_i, \ldots, v_t=v_1, \ldots, v_{i-1})
    \]
    содержит на 1 вершину больше чем путь $\tau$, что противоречит выбору $\tau$ как пути с самым большим количеством вершин. Значит рёбер вида ($u, v_i$), не входящих в $\tau$ нет.
    \end{itemize}
    
    (то есть мы доказали, что в каждой вершине цикла $\tau$ использованы все рёбра, а этот случай мы уже рассмотрели)
\end{itemize}

\end{proof}

\end{document}